\chapter{Modellbildung}
\label{sec: Kapitel 2}

\section{Berechnung der Bewegungsgleichung}
Die kinetische Energie des Hubwerksgetriebes schreibt sich als
\[
T
= \frac{1}{2} J_2 \dot{\varphi}_2^{\,2}
+ \frac{1}{2} J_3 \dot{\varphi}_3^{\,2}
+ \frac{1}{2} J_4 \dot{\varphi}_4^{\,2}
+ \frac{1}{2} m_5 \dot{y}_5^{\,2}
\]

Aus der Forderung, dass die überstrichenen Bogenlängen beim Abwälzvorgang
für beide involvierten Körper gleich sind, ergibt sich
\[
\varphi_2 r_2 = - \varphi_3 r_{32}
\;\Rightarrow\;
\varphi_3 = - \varphi_2 \frac{r_2}{r_{32}}
\]

\[
\varphi_3 r_{34} = - \varphi_4 r_{43}
\;\Rightarrow\;
\varphi_4 = \varphi_2 \frac{r_2}{r_{32}} \frac{r_{34}}{r_{43}}
\]

\[
y_{55} = \varphi_4 r_{45} - l_0
\;\Rightarrow\;
y_{55} = \varphi_2 \frac{r_2}{r_{32}} \frac{r_{34}}{r_{43}} r_{45} - l_0
\]

Das Einsetzen des Quadrates der zeitlichen Ableitungen in die kinetische
Energie liefert das reduzierte Massenträgheitsmoment \( J_{\text{red}} \).

\[
T
= \frac{1}{2} \dot{\varphi}_2^{\,2}
\left(
J_2
+ J_3 \left( \frac{r_2}{r_{32}} \right)^2
+ J_4 \left( \frac{r_2 r_{34}}{r_{32} r_{43}} \right)^2
+ m_5 \left( \frac{r_2 r_{34} r_{45}}{r_{32} r_{43}} \right)^2
\right)
= \frac{1}{2} \dot{\varphi}_2^{\,2} J_{\text{red}}
\]

Die potentielle Energie ergibt sich zu
\[
V
= m_5 g
\left(
\varphi_2 \frac{r_2 r_{34} r_{45}}{r_{32} r_{43}}
\right)
\]

und unter Anwendung der Lagrange-Gleichung ergibt sich mit dem
Antriebsmoment \( M_2 \) die Bewegungsgleichung
\[
J_{\text{red}} \ddot{\varphi}_2
+ m_5 g \frac{r_2 r_{34} r_{45}}{r_{32} r_{43}}
= M_2
\]

Simulink arbeitet am einfachsten mit expliziten Integrationsketten:
\[
\ddot{\varphi}_2
= \frac{1}{J_{\mathrm{red}}}
\left(
M_2
- m_5 g \frac{r_2 r_{34} r_{45}}{r_{32} r_{43}}
\right)
\]

\section{Blockschaltbild}
\begin{center}
\begin{tikzpicture}[
    >=stealth,
    block/.style={draw, rectangle, minimum width=2.6cm, minimum height=1.2cm, align=center},
    sum/.style={draw, circle, minimum size=1.2cm},
    int/.style={draw, rectangle, minimum width=1.2cm, minimum height=1.2cm}
]

% Knoten
\node[sum]   (sum)  at (0,0) {};
\node[block] (gain) at (3.2,0) {$\dfrac{1}{J_{\mathrm{red}}}$};
\node[int]   (int1) at (6.2,0) {$\displaystyle \int$};
\node[int]   (int2) at (8.6,0) {$\displaystyle \int$};

% Beschriftungen Summationsknoten
\node at (-0.2,0.35) {$+$};
\node at (-0.2,-0.35) {$-$};

% Eingänge
\draw[->] (-2.2,0.6) node[left] {$M_2$} -- (sum.west |- 0,0.6);
\draw[->] (-2.2,-0.6) node[left] {$C_g$} -- (sum.west |- 0,-0.6);

% Hauptpfad
\draw[->] (sum) -- (gain);
\draw[->] (gain) -- (int1);
\draw[->] (int1) -- (int2);
\draw[->] (int2) -- (10.2,0) node[right] {$\varphi_2$};

% Abzweig auf Winkelgeschwindigkeit
\draw[->] (int1.south) |- (6.2,-1.8) -- (10.2,-1.8) node[right] {$\dot{\varphi}_2$};

\end{tikzpicture}
\end{center}

