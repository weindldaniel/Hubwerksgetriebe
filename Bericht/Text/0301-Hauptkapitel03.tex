\chapter{Implementierung der Regelung}
\label{sec: Kapitel 2}

\section{Umsetzung in Csharp}

Die Bewegungsgleichung aus Gleichung (2.1) wird an dieser Stelle in einer SContinuousFunction umgesetzt. Hierzu wird das entsprechende Blockschaltbild nachgebildet, wie im nachfolgenden Code dargestellt.

\begin{lstlisting}[style=csharpstyle]
    public class Hubwerk : Example
    {
        public Hubwerk()
        {
            double _r2 = 2;
            double _r32 = 5;
            double _r34 = 2;
            double _r43 = 4;
            double _r45 = 2;
            double m = 25;
            double g = 9.81;
            double jRed = 1.6;
            double M2 = 50;
            double a = (_r2 * _r34 * _r45) / (_r32 * _r43);
            
            Block constantM = new ConstantBlock("Constant", (M2/jRed));
            Block constantC = new ConstantBlock("Constant", ((m*g*a)/jRed));
            Block sub = new SubtractBlock("Sub");
            Block integrate1 = new IntegrateBlock("Integrate1", 0);
            Block integrate2 = new IntegrateBlock("Integrate2", 0);
            Block recordphi = new RecordBlock("Phi");
            Block recordphidot = new RecordBlock("Winkelgeschwindigkeit");
            Block recordphidotdot = new RecordBlock("Winkelbeschleunigung");
            
            Model.AddBlock(constantM);
            Model.AddBlock(constantC);
            Model.AddBlock(sub);
            Model.AddBlock(integrate1);
            Model.AddBlock(integrate2);
            Model.AddBlock(recordphi);
            Model.AddBlock(recordphidot);
            Model.AddBlock(recordphidotdot);
            
            Model.AddConnection(constantM, 0, sub, 0);
            Model.AddConnection(constantC, 0, sub, 1);
            Model.AddConnection(sub, 0, integrate1, 0);
            Model.AddConnection(integrate1, 0, integrate2, 0);
            Model.AddConnection(integrate2, 0, recordphi, 0);            
            Model.AddConnection(integrate1, 0, recordphidot, 0);
            Model.AddConnection(sub, 0, recordphidotdot, 0);
        }
    }
\end{lstlisting}

Wir lassen das Example Hubwerk builden:
\begin{figure}[h]
	\centering
	\includegraphics[width=0.80\textwidth]{/Users/danielweindl/_source/Repositorys/SSI_Hubwerksgetriebe/Bericht/Grafik/HubwerkSFUnc.png}
	\caption{}
\end{figure}

\section{Extraktion von phi}
Nach mehreren Tests wurde die Plausibilität der Modellbildung als bestätigt angesehen. Im nächsten Schritt soll die Liste der Variable phi in eine CSV-Datei exportiert werden.

Den Code dafür binden wir direkt nach dem Solver ein.

\begin{lstlisting}[style=csharpstyle]

namespace SFunctionContinuous
{
    /// <summary>
    /// Interaction logic for MainWindow.xaml
    /// </summary>
    public partial class MainWindow : Window
    {
        public MainWindow()
        {
            InitializeComponent();

            // Modell erstellen und loesen

            //Example example = new PosServo(); 
            Example example = new Hubwerk();

            try
            {
                Solver solution = new EulerExplicitSolver(example.Model);
                solution.Solve(example.TimeStepMax, example.TimeMax);
            }
            catch (Exception e)
            {
                MessageBox.Show(e.Message);
            }
            // ================================
            // CSV-EXPORT VON PHI (HIER!)
            // ================================

            RecordBlock phiBlock = example.Model.Blocks
                .OfType<RecordBlock>()
                .First(b => b.Name == "Phi");

            string filePath = @"Z:\_source\Repositorys\SSI_Hubwerksgetriebe\phi2.csv";

            CultureInfo culture = CultureInfo.InvariantCulture;
            StringBuilder sb = new StringBuilder();

            sb.AppendLine("phi");

            foreach ((_, double phi) in phiBlock.Data)
            {
                sb.AppendLine(phi.ToString(culture));
            }

            File.WriteAllText(filePath, sb.ToString());
            
            // Graph-Visualisierung erstellen

            Graph graph = new Graph();

            foreach (Block f in example.Model.Blocks)
            {
                graph.AddNode($"{f.GetHashCode()}").LabelText = f.ToString();
            }
            foreach (Connection c in example.Model.Connections)
            {
                graph.AddEdge($"{c.Source.GetHashCode()}", c.ToString(), $"{c.Target.GetHashCode()}");
            }

            Graph.Graph = graph;

            // Chart-Visualisierung erstellen

            foreach (Block f in example.Model.Blocks)
            {
                if (f is RecordBlock)
                {
                    RecordBlock r = (RecordBlock)f;

                    double[] t = new double[r.Data.Count];
                    double[] u = new double[r.Data.Count];

                    for (int i = 0; i < r.Data.Count; i++)
                    {
                        (double ti, double tu) = r.Data[i];

                        t[i] = ti;
                        u[i] = tu;
                    }

                    Scatter scatter = Chart.Plot.Add.Scatter(t, u);

                    scatter.LegendText = f.Name;
                }
            }
        }
    }
}
\end{lstlisting}

Jetzt muss die Visualisierung nur noch auf die Liste zugreifen.
