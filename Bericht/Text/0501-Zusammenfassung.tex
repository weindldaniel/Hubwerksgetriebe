\chapter{Fazit}
\label{sec: Zusammenfassung}

Durch die konsequente Trennung von Berechnung und Visualisierung war es möglich, beide Entwicklungsbereiche parallel zu beginnen und weitgehend unabhängig voneinander weiterzuentwickeln. Als einzige Schnittstelle zwischen diesen beiden Bereichen dient die CSV-Datei, über welche die berechneten Daten an die Visualisierung übergeben werden. Diese klare Trennung erwies sich als vorteilhaft für Struktur, Wartbarkeit und Parallelisierung der Entwicklungsarbeit.

Die Umsetzung stellte sich insgesamt als anspruchsvoller heraus als ursprünglich angenommen. Zwar wurde für die Visualisierung KI-Unterstützung herangezogen, dennoch waren zahlreiche manuelle Anpassungen notwendig. Insbesondere zeigte sich, dass ein grundlegendes Verständnis des erzeugten Codes sowie eine im Vorfeld durchdachte Softwarearchitektur unerlässlich sind, um die Ergebnisse zielgerichtet weiterentwickeln zu können.

Zur besseren Nachvollziehbarkeit und Überprüfung der Rotationsbewegungen wurden an den Rollen gezielt Markierungen angebracht. Dadurch konnten die Bewegungsabläufe visuell überprüft und mit den berechneten Werten abgeglichen werden.

Die Kopplung von Logik und Visualisierung funktionierte insgesamt sehr gut, was maßgeblich darauf zurückzuführen ist, dass bereits im Vorfeld klare konzeptionelle Überlegungen zur Schnittstelle und zum Datenfluss angestellt wurden.

Abschließend lässt sich festhalten, dass das Arbeiten über mehrere Disziplinen hinweg sowie deren Zusammenführung durch Code eine besonders anspruchsvolle, zugleich jedoch sehr interessante und lehrreiche Tätigkeit darstellt.
