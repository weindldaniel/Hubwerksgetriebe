\chapter{Einleitung}
\label{sec: Einleitung}

\section{Aufgabenstellung}
Das Hubwerksgetriebe aus dem Skript MMB4 von Dr. Witteeveen soll mit einem Csharp Programm simuliert werden. 
Dazu sind die einzelnen Komponenten des Getriebes zu modellieren und in einem Gesamtmodell zu verknüpfen.
Die Simulation soll es ermöglichen, verschiedene Lastfälle durchzuspielen und die Auswirkungen auf die einzelnen
Komponenten zu beobachten.

\begin{figure}[h]
	\centering
	\includegraphics[width=0.80\textwidth]{/Users/danielweindl/_source/Repositorys/SSI_Hubwerksgetriebe/Bericht/Grafik/1.2.2-Hubwerksgetriebe.png}
	\caption{Aufgabe: Hubwerksgetriebe}
\end{figure}

\section{Aufgabenverteilung}
\begin{itemize}

	\item Visualisierung des Hubwerkgetriebes - DW
	\item Modellierung der Bewegungsgleichung - TP
	
	\item Implementierung der Regelung - TP und DW
	\item Simulation des Gesamtsystems - TP und DW

\end{itemize}

\section{Vorgehensweise}
Die Idee besteht darin, mithilfe einer SFunctionContinuous eine Regelung zu simulieren. Die dabei berechneten Werte der Größe phi sollen anschließend in eine CSV-Datei geschrieben werden, welche in weiterer Folge zur Visualisierung in OpenGL verwendet wird.