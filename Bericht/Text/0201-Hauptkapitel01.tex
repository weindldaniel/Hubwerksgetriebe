\chapter{Visualisierung}
\label{sec: Kapitel 1}

Das Hubwerksgetriebe soll so wie in der Aufgabenstellung abgebildet werden.

\section{Erstellung der Rollen}

Jede Rolle wird nach folgedenm Muster aufgebaut.
Den größten Teil des Codes haben wir in die Hilfsmethode "DraxDiskXy ausgelagert.
\begin{lstlisting}[style=csharpstyle]
    SetMaterial(gl, 0f, 0f, 0.5f);
    gl.PushMatrix();
    gl.Translate(_x2, 0, 0);
    gl.Rotate(_phi2 * 180.0 / Math.PI, 0, 0, 1);
    DrawDiskXy(gl,_r2,_thickness2,_segments,0,0,0,DiskMarkerColor.Red,DiskMarkerDirection.PositiveX);
    gl.PopMatrix();  
\end{lstlisting}

\section{Erstellung von Seil und Punktmaße}
Die Erstellung der Punktmaße wurde auch in eine Hilfsmethode ausgelagert.
\begin{lstlisting}[style=csharpstyle]
    // ---------------- Seil ----------------
    gl.Disable(OpenGL.GL_LIGHTING);
    gl.Color(0, 0, 0);
    gl.LineWidth(3.0f);
    gl.Begin(OpenGL.GL_LINES);
    gl.Vertex(x4+_r45, 0, ropeZ);   // Austritt an der Rolle
    gl.Vertex(x4+_r45, ropeY, ropeZ);       // bewegte Masse
    gl.End();
    gl.Enable(OpenGL.GL_LIGHTING);
    // ---------------- Punktmasse ----------------
    SetMaterial(gl, 0, 0, 0);
    DrawSphere(gl, 0.5, 30, 30, x4+_r45, ropeY - 0.4, ropeZ);
\end{lstlisting}

\section{Ergebnis}
Die Markierungen auf den Rollen lassen gut nachvollziehen dass die Rotationen richtig übertragen werden.
\begin{figure}[h]
	\centering
	\includegraphics[width=0.80\textwidth]{/Users/danielweindl/_source/Repositorys/SSI_Hubwerksgetriebe/Bericht/Grafik/Visu.png}
	\caption{Ergebnis}
\end{figure}
