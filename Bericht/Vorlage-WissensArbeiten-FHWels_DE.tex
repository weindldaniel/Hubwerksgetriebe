%%++++++++++++++++++++++++++++++++++++++++++++++++++++++++++++++++++++++++++++
%%            INFORMATION
%%============================================================================
%%            Version
%%----------------------------------------------------------------------------
\def\CurrentVersionMain{\textit{V2.0}}
%%============================================================================
%%            Dokumentenklasse und Präambel
%%----------------------------------------------------------------------------
\documentclass[12pt,a4paper,oneside,bibliography=totocnumbered,toc=listofnumbered]{scrbook}
\usepackage[english,ngerman,naustrian]{babel}
\usepackage{./packages/WissensArbeitenFHWels}
\usepackage{./packages/WissensArbeitenFHWels-FrontseitenDE}     
%%++++++++++++++++++++++++++++++++++++++++++++++++++++++++++++++++++++++++++++
%%
%%++++++++++++++++++++++++++++++++++++++++++++++++++++++++++++++++++++++++++++
%%            DATEN/EINSTELLUNGEN für die WISSENSCHAFTLICHE ARBEIT
%%============================================================================
%%            Daten für die Arbeit
%%----------------------------------------------------------------------------
%% Titel der Arbeit:
\title{Simulation eines Hubwerksgetriebe}
%%
%% Autor der Arbeit:
\author{Peböck Thomas - Weindl Daniel}
%%
%% Studiengang:
\studiengang{Automatisierungstechnik}
%%
%% Typ des Dokuments:
\typDokument{Projektbericht}
% -) "Bachelorarbeit"
% -) "Masterarbeit"
% -) "Projektbericht"
%%
%% Art des Studiengangs:
\artStudiengang{Bachelorstudiengang}
% -) "Bachelorstudiengang"
% -) "Masterstudiengang"
%%
%% Zu erlangender akademischer Grad:
\akademGrad{Bachelor of Science in Engineering (BSc)}
% -) "Bachelor of Science in Engineering (BSc)"
% -) "Master of Science in Engineering (MSc)"
% -) "Diplom-Ingenieurin für technisch-wissenschaftliche Berufe (Dipl.-Ing.)"
% -) "Diplom-Ingenieur für technisch-wissenschaftliche Berufe (Dipl.-Ing.)"
%%
%% Ort der Fakultät:
\studienort{Wels}
%%
%% Abgabemonat und Jahr:
\abgabemonat{Jänner}
\abgabejahr{2026}
%%
%% Betreuer der Arbeit:
\betreuer{Dr. Georg Hackenberg}
%%
%% Stichwörter für die Arbeit:
%% (werden in die Metadaten des PDF's geschrieben)
\stichwoerter{Stichwort 1, Stichwort 2 Stichwort 3...}
%%
%% Unterschrift anzeigen: (ACHTUNG: Syntax wichtig!!!)
\ShowSignature{0mm}{22mm}{0.5}{No}		% Zeigt die Unterschrift auf der Eidest. Erklärung
% Syntax: {x-Position}{y-Position}{Skalierungsfaktor}{'Anzeigen:' Yes / No}
% Beispiel: \ShowSignature{0mm}{22mm}{0.5}{Yes}
% Datei mit dem Namen "Signature.png" muss im Grafik-Ordner vorhanden sein.
%%
%%	Eingefärbung der Links festlegen: (ACHTUNG: Syntax wichtig!!!)
\ColoredLinks{No}		% Eingefärbet Links(Abbildung, URL's, Literatur)
% Syntax: Yes / No
%%
%%============================================================================
%%            Einstellungen:
%%----------------------------------------------------------------------------
\Style{FHWels}				% Grundlegender Stil (nur 'FHWels' möglich)
\BibStyle{FHWelsNumericBrackets}	% Zitier-Stil auf Basis von 'BibLatex'
% Syntax:	-) FHWelsNumericBrackets (z.B.: [1])
%			-) FHWelsAlphabeticBrackets (z.B. [Meier, 2011])
\addbibresource{Bibliography.bib}	% Bibliography - Datei(en)
\graphicspath{/Users/danielweindl/_source/Repositorys/SSI_Hubwerksgetriebe/Bericht/Grafik}	% Festlegung des Unterordners für Abbildungen
\TabContent{1.4cm}			% Tabulatoreinrückung für das Inhaltsverzeichnis	
%%++++++++++++++++++++++++++++++++++++++++++++++++++++++++++++++++++++++++++++
%%
%%++++++++++++++++++++++++++++++++++++++++++++++++++++++++++++++++++++++++++++
%%            BEGINN der ARBEIT
%%============================================================================
\begin{document}
\selectlanguage{naustrian}	% Deutsche Sprache (Österreichpaket)	
%%============================================================================
%%            Titelseite & Eidesstattliche Erklärung
%%----------------------------------------------------------------------------
%% Titelseite
	\frontmatter				% römische Seitennummerierung				
	\pagenumbering{Roman}		% Großbuchstaben
	\titelseite
%%	
%%============================================================================
%%            Vorwort & weitere Dokumente
%%----------------------------------------------------------------------------
% %% Vorwort, Kurzfassung/Abstract, Inhaltsverzeichnis
% 	\begingroup
% 		\pagestyle{plain}				%...keine Kopfzeile
% 		\renewcommand{\chapterpagestyle}{plain}%
% 		\chapter{Vorwort/Danksagung}

Das Vorwort enthält einen Kurzhinweis auf die Themenstellung und optional – unter Berücksichtigung der geltenden Datenschutzbestimmungen – Dankesworte an Firmen, Berater*innen, Betreuer*innen, Familienangehörige usw. für die Unterstützung bei der Durchführung der Arbeit.

% \section{Verwendung der Vorlage}
% Dieses Dokument ist als Vorlage gedacht und enthält optionale Verzeichnisse. Nicht benötigte Verzeichnisse (Abkürzungsverzeichnis, Fachwortverzeichnis, Abbildungsverzeichnis, Tabellenverzeichnis und Formelverzeichnis) können in Absprache mit Ihrem*Ihrer Betreuer*in gelöscht werden. Bitte besprechen Sie im Vorfeld mit Ihrem*Ihrer Betreuer*in, welche Formatierung, Zitierweise, Kapitelstruktur etc. Sie verwenden sollen. Eventuelle studiengangsspezifische Regelungen sind ggf. einem gesonderten Dokument zu entnehmen. Bitte sprechen Sie diesbezüglich Ihren*Ihre Betreuer*in an.
% \vspace{6pt}\\
% Die Erklärungen in dieser Vorlage dienen nur zu Ihrer Information. Ersetzen Sie sie nach sorgfältiger Lektüre und Beachtung durch Ihren eigenen Text! Löschen Sie unzutreffende Informationen. Bitte beachten Sie, dass die automatische Nummerierung im Gesamtdokument erst dann korrekt dargestellt wird, wenn Sie die nur zur Erklärung der Vorlage dienenden Unterkapitel des Vorwortes gelöscht haben.
% \vspace{6pt}\\
% Für eine sichere Aktualisierung des Inhaltsverzeichnisses empfiehlt es sich, das Dokument immer zweimal zu kompilieren.
% \vspace{6pt}\\
% \textcolor{red}{\textbf{Die vorliegende Vorlage wurde nach bestem Wissen und Gewissen erstellt. Ihre Verwendung geschieht jedoch auf eigene Gefahr und Haftung der Nutzer*innen für etwaige Datenverluste, für Computerprobleme oder finanzielle oder sächliche Schäden, die durch die Verwendung dieser Vorlage entstehen können. Die Autor*innen  der vorliegenden Formatvorlage und die FH Oberösterreich übernehmen keine Verantwortung für die korrekte Funktionsweise dieser Vorlage.}}

% \newpage
% \subsection{Vorkenntnisse \LaTeX}
% Für die Verwendung dieser Vorlage werden Vorkenntnisse mit dem Softwarepaket {\LaTeX} vorausgesetzt. Die angebotene Vorlage wurde mit dem Programm \textbf{TeXstudio} (\url{www.texstudio.org}) und der Windows-Distribution \textbf{MiKTEX} (\url{www.miktex.org}) erstellt. Bei der Verwendung anderer Software kann, muss es aber nicht, zu Problemen kommen. Die Verwendung des Apple-Betriebssystems wird nicht unterstützt.\\
% Es werden in dieser Vorlage keine Befehle erklärt \bzw werden die verwendeten Style\hbox{-}Pakete nicht weiter betrachtet. Umfangreiche Hilfe wird im Internet in zahlreichen Foren und auf Webseiten angeboten.

% \section{Überschriften}
% Beachten Sie in Bezug auf Überschriften die folgenden Punkte zur Formulierung und Gliederung:
% \begin{itemize}
% 	\item Keine unbekannten Abkürzungen in Überschriften
% 	\item Keine Artikel (\textit{der, die, das}) am Anfang von Überschriften
% 	\item Kein Punkt am Ende von Überschriften
% 	\item Achten Sie darauf, dass keine Überschrift – ohne den zugehörigen Text – die letzte Zeile einer Seite bildet; es sollte auch keine Seite mit einer Einzelzeile beginnen, die zum letzten Absatz der vorhergehenden Seite gehört.
% 	\item Detaillierung entspricht den Schwerpunkten bzw. der Gewichtung der Arbeit
% 	\item Immer mindestens zwei Unterpunkte zu einem Gliederungspunkt, ansonsten nicht gliedern
% 	\item Verwenden Sie maximal 5 Gliederungsebenen	
% \end{itemize}
% Aufgrund der automatischen Gliederung von {\LaTeX} können die Überschriften hier nicht dargestellt, sondern nur beschrieben werden:
% \begin{enumerate}
% 	\item	Gliederungsebene: \verb|\chapter{...}|-Befehl
% 	\item	Gliederungsebene: \verb|\section{...}|-Befehl
% 	\item	Gliederungsebene: \verb|\subsection{...}|-Befehl
% 	\item	Gliederungsebene: \verb|\subsubsection{...}|-Befehl
% 	\item	Gliederungsebene: \verb|\paragraph{...}|-Befehl
% \end{enumerate}

% \newpage
% \section{Abkürzungen}
% Die Zeichenabstände werden von {\LaTeX} teilweise automatisch gesetzt. Damit übliche Abkürzungen richtig dargestellt werden, wurden diese mit Befehlen vordefiniert.

% \begin{multicols}{2}	
% 	\begin{itemize}
% 		\item Befehl: \verb|\bzw|\dotfill\bzw
% 		\item Befehl: \verb|\bzgl|\dotfill\bzgl
% 		\item Befehl: \verb|\ca|\dotfill\ca
% 		\item Befehl: \verb|\dah|\dotfill\dah
% 		\item Befehl: \verb|\Dah|\dotfill\Dah
% 		\item Befehl: \verb|\ds|\dotfill\ds
% 		\item Befehl: \verb|\evtl|\dotfill\evtl
% 		\item Befehl: \verb|\ua|\dotfill\ua
% 		\item Befehl: \verb|\Ua|\dotfill\Ua
% 		\item Befehl: \verb|\usw|\dotfill\usw
% 		\item Befehl: \verb|\va|\dotfill\va
% 		\item Befehl: \verb|\vgl|\dotfill\vgl
% 		\item Befehl: \verb|\zB|\dotfill\zB
% 		\item Befehl: \verb|\ZB|\dotfill\ZB
% 		\item Befehl: \verb|\sa|\dotfill\sa
% 		\item Befehl: \verb|\ia|\dotfill\ia
% 		\item Befehl: \verb|\su|\dotfill\su
% 		\item Befehl: \verb|\uvm|\dotfill\uvm
% 		\item Befehl: \verb|\uva|\dotfill\uva
% 		\item Befehl: \verb|\uae|\dotfill\uae
% 		\item Befehl: \verb|\mn|\dotfill\mn
% 		\item Befehl: \verb|\mx|\dotfill\mx
% 		\item Befehl: \verb|\etc|\dotfill\etc
% 		\item Befehl: \verb|\ggf|\dotfill\ggf
% 	\end{itemize}
% \end{multicols}

% \newpage
% \section{Grafiken}
% \label{sec: Grafik}
% Damit die Grafiken richtig nummeriert werden und auch eine Referenz auf sie gesetzt werden kann, muss die Bild-Umgebung verwendet werden.\\
% Mit folgendem Beispielcode kann eine Abbildung eingefügt werden:
% \begin{figure}[h]
% 	\centering
% 	\includegraphics[width=0.50\textwidth]{FH.png}
% 	\caption{Beispiel für eine Abbildung}
% 	\label{fig: Beispielabbildung}
% \end{figure}\\
% Zwei Abbildungen nebeneinander, mit einer Beschriftung, können wie folgt eingefügt werden:
% \begin{figure}[htp!]
% 	\begin{minipage}[t]{0.49\textwidth}
% 		\centering
% 		\includegraphics[width=0.49\textwidth]{FH.png}
% 	\end{minipage}
% 	\hfill
% 	\begin{minipage}[t]{0.49\textwidth}
% 		\centering
% 		\includegraphics[width=0.49\textwidth]{FH.png}
% 	\end{minipage}
% 	\caption{Beispiel für 2 Abbildungen nebeneinander, 1 Beschriftung}
% 	\label{fig: Beispielabbildung-2}
% \end{figure}\\
% Zwei Abbildungen nebeneinander, mit getrennter Beschriftung, können wie folgt eingefügt werden:
% \begin{figure}[htp!]
% 	\begin{minipage}[t]{0.49\textwidth}
% 		\centering
% 		\includegraphics[width=0.49\textwidth]{FH.png}
% 		\caption{Beispiel für 2 Abbildungen nebeneinander, 2 Beschriftungen}
% 		\label{fig: Beispielabbildung-3}
% 	\end{minipage}
% 	\hfill
% 	\begin{minipage}[t]{0.49\textwidth}
% 		\centering
% 		\includegraphics[width=0.49\textwidth]{FH.png}
% 		\caption{Beispiel für 2 Abbildungen nebeneinander, 2 Beschriftungen}
% 		\label{fig: Beispielabbildung-4}
% 	\end{minipage}
% \end{figure}\\
% Beachten Sie folgende Punkte bei der Verwendung von Abbildungen:
% \begin{itemize}
% 	\item	Abbildungen sollten im Text referenziert \bzw mit dem Text in Bezug gebracht werden, ansonsten sind sie überflüssig. Verwenden Sie dazu den Befehl \verb|\ref{}|
% 	[siehe Abb.~\ref{fig: Beispielabbildung}] \\
% 	Eine zusätzliche Seitenreferenz erhält man mit dem Befehl \verb|\pageref{}| \newline [siehe Abb.~\ref{fig: Beispielabbildung} auf S.~\pageref{fig: Beispielabbildung}]
% 	\item Nicht jede Grafik muss in den Text. Auswertungen, Tabellenwerke, Back-up-Informationen \etc können im Anhang zusammengefasst werden. Jedoch sollte ein Hinweis im Text enthalten sein.
% 	\item Ergänzen Sie \ggf eine Legende in den Abbildungen und die erforderlichen Quellenangaben.
% \end{itemize}

% \newpage
% \section{Tabellen}
% Tabellen sollten im Text referenziert \bzw mit dem Text in Bezug gebracht werden [siehe Tabelle \ref{tab: Testtabelle} auf S.~\pageref{tab: Testtabelle}].\\
% Für die Tabellen gibt es eine eigene Tabellen-Umgebung damit die Nummerierung an die richtige Position gesetzt wird und auch auf sie referenziert werden kann.\\
% Die Tabelle selbst kann auf zwei unterschiedlichen Varianten eingefügt werden.
% \subsection{Als Bild}
% In der Tabellen-Umgebung kann ein Bild einer Tabelle eingefügt werden.
% \begin{table}[h]
% 	\centering
% 	\caption{Testtabelle}
% 	\includegraphics[scale=0.6]{0000_Testtabelle.png}
% 	\label{tab: Testtabelle} 
% \end{table}
% \subsection{Als Tabelle}
% Es können auch direkt im {\LaTeX} Tabellen erstellt werden.
% \begin{table}[h]
% 	\centering
% 	\caption{Testtabelle 2}
% 	\begin{tabular}{c|c||c|}
% 		Spaltenüberschrift & Spaltenüberschrift & Spaltenüberschrift \\
% 		\midrule[2pt]
% 		1 & Test-1 & Test-2\\ 
% 		\hline 2 & 0 & 0  \\ 
% 		3 & 0 & 0  \\ 
% 		4 & 0 & 1  \\
% 		\hline 
% 	\end{tabular}
% 	\label{tab: Testtabelle-2} 
% \end{table}

% \section{Formeln und Gleichungen}
% Formeln und Gleichungen sind grundsätzlich in einen Satz integriert. Sie können direkt in den Fließtext, zum Beispiel in der Form $e=m\cdot c^2$, oder abgesetzt in eine eigene Zeile/einen eigenen Absatz geschrieben werden.
% Im letzteren Fall kann eine Formel, zum Beispiel
% %
% \begin{align}
% 	e=m\cdot c^2,
% 	\label{eq: Formel1}
% \end{align}
% %
% mit einer Nummer versehen werden, auf die im Text an späterer Stelle verwiesen wird (siehe Gleichung~\ref{eq: Formel1} auf S.~\pageref{eq: Formel1}).
% \vspace{6pt}\\
% Weitere Beispiele für die Einbettung von Formeln:
% \\
% Das Newtonsche Gesetz $F=ma$ ist eines der wichtigsten Naturgesetze der Physik.
% \\
% Die allgemeine Form einer Fourier-Reihe lautet
% %
% \begin{align}
% 	f(x)=a_0 +\sum_{n=1}^\infty \left( a_n \cos\frac{n\pi x}{L} + b_n \sin\frac{n\pi x}{L} \right) .
% 	\label{eq: Formel2}
% \end{align}
% %
% Darin bezeichnet man $a_n$ und $b_n$ als Fourier-Koeffizienten. Man beachte dabei die Satzzeichen
% in Gleichung~\ref{eq: Formel2} und in Gleichung~\ref{eq: Formel1}!

% \section{Zitieren}
% Beim Zitieren können Sie eine der folgenden Vorgehensweisen frei wählen:
% \begin{itemize}
% 	\item rein manuelles Zitieren;
% 	\item Zitieren mithilfe eines Literaturverwaltungsprogramms.
% \end{itemize}
% Entscheidend ist, dass Sie die Quellenangaben und das Literaturverzeichnis Ihrer wissenschaftlichen Arbeit richtig gestalten, \dah die Quellenangaben gemäß Abschnitt \ref{sec: ZitierStil} auf S. \pageref{sec: ZitierStil} und das Literaturverzeichnis gemäß Abschnitt \ref{sec: Bibliography} auf S. \pageref{sec: Bibliography}.\\
% Stimmen Sie bereits zu Beginn Ihrer Arbeit die Zitierweise und etwaige zu verwendende Zitiervorlagen mit Ihrem*Ihrer Betreuer*in ab. Die Verwendung von Zitiervorlagen erfordert unter Umständen manuelle Anpassungen vor Abgabe Ihrer wissenschaftlichen Arbeit.\\
% Für das Literaturverzeichnis in {\LaTeX} wird eine *.bib-Datei benötigt. Es gibt mehrere Möglichkeiten, diese Datei zu erstellen (manuell, JabRef, \dots).

% \subsection{Wichtige Hinweise zu Zitaten}
% \begin{itemize}
% 	\item	Es werden wörtliche Zitate (grundsätzlich in Anführungszeichen) und sinngemäße Zitate (ohne Anführungszeichen) differenziert.
% 	\begin{addmargin}[10pt]{0pt}
% 		{\footnotesize Jedes Zitat muss überprüfbar und einwandfrei nachvollziehbar sein. Einwandfreies Zitieren ist Ausdruck wissenschaftlicher Sorgfalt. Übernommenes fremdes Gedankengut ist in jedem Fall – egal ob als wörtliches oder sinngemäßes Zitat – als solches kenntlich zu machen.\footnote{Karmasin/Ribing, 2019, S. 114.}}
% 	\end{addmargin}
	
% 	\item	Integrieren Sie kurze wörtliche Zitate in den Fließtext. Verwenden Sie bei Zitaten, die in Standardformatierung länger als drei Zeilen sind, folgende Vorlage. In diesem Fall verwenden Sie keine Anführungszeichen:
% 	\begin{addmargin}[10pt]{0pt}
% 		{\footnotesize Lorem ipsum dolor sit amet, consectetuer adipiscing elit. Maecenas porttitor congue massa. Fusce posuere, magna sed pulvinar ultricies, purus lectus malesuada libero, sit amet commodo magna eros quis urna. Nunc viverra imperdiet enim. Fusce est. Vivamus a tellus. Pellentesque habitant morbi tristique senectus [\dots] fames ac turpis egestas. Proin pharetra nonummy pede. \cite[S. 6]{Meier:Globalisierung}}
% 	\end{addmargin}
	
% 	\item	„Ein Plagiat ist nicht nur ein wörtliches Zitat ohne Anführungszeichen, sondern auch ein sinngemäßes indirektes Zitat, das den Anschein erweckt, es sei aus eigenen Erkenntnissen entstanden.“\footnote{Ebd.}
	
% 	\item	Wörtliche Zitate sollten nur im Ausnahmefall unmittelbar einer Überschrift folgen.
	
% 	\item	Bei Zitaten dürfen keine inhaltlichen Veränderungen vorgenommen werden.
	
% 	\item	Alle Aussagen ohne Quellenangabe stellen entweder eigene Standpunkte, eigene Erkenntnisse oder generell anerkannte Tatsachen dar.
	
% 	\item	Übersetzungen müssen in der Fußnote kenntlich gemacht werden.
	
% 	\item	Zahlen haben nur dann einen wissenschaftlichen Wert, wenn sie nachprüfbar sind. Daher ist immer eine Quellenangabe bei Zahlenangaben erforderlich.
	
% \end{itemize}

% \subsection{Wörtliche und sinngemäße Zitate}
% \begin{itemize}
% 	\item	Wörtlich übernommene Texte sind immer in Anführungszeichen zu setzen, außer es handelt sich um ein Blockzitat [siehe oben]. Verwenden Sie stets die deutschen doppelten Anführungszeichen („ “) und nicht die englischen (“ ”) Anführungszeichen zur Kennzeichnung eines Zitates. Enthält ein deutsches Zitat bereits Anführungszeichen, so ändern Sie diese in einfache deutsche Anführungszeichen (‚  ‘) um:
% 	\begin{addmargin}[10pt]{0pt}
% 		\textcolor{red}{Müller schreibt: „In früheren Zeiten hatte die Obrigkeit die Finanzgewalt in ihrer ‚festen‘ Hand“ [Müller, 2010, S. 5].}
% 	\end{addmargin}
% 	\item	Eigene Ergänzungen oder Veränderungen sind in eckige Klammern zu setzen. Dabei darf der Inhalt des Zitats nicht verändert werden:
% 	\begin{addmargin}[10pt]{0pt}
% 		\textcolor{red}{Nach Auffassung von Müller „[hatte] in früheren Zeiten die Obrigkeit die Finanzgewalt [\dots]“ [Müller, 2010, S. 5].}
% 	\end{addmargin}
% 	\item	Auslassungen eines Wortes oder mehrerer Wörter sind durch 3 Punkte in eckigen Klammern anzudeuten. Die Auslassungspunkte sind über den Befehl \verb|\dots| zu erzeugen (und nicht durch drei Satzpunkte!):
% 	\begin{addmargin}[10pt]{0pt}
% 		\textcolor{red}{„In früheren Zeiten hatte die Obrigkeit […]“ [Müller, 2010, S. 5].}
% 	\end{addmargin}
% 	\item	Bei sinngemäßer Wiedergabe erfolgt lediglich die Angabe der Quelle ohne Anführungszeichen:
% 	\begin{addmargin}[10pt]{0pt}
% 		\textcolor{red}{Nach Auffassung von Müller besaß die Obrigkeit Kontrolle über die Finanzen [vgl. Müller, 2010, S. 5].}
% 	\end{addmargin}
% 	\item	Ein wiederholtes Zitieren derselben Quelle kann mit \textsl{ebenda} durchgeführt werden. Es muss dabei ein direkter Quellenbezug gegeben sein:
% 	\begin{addmargin}[10pt]{0pt}
% 		\textcolor{red}{Nach Auffassung von Müller besaß die Obrigkeit Kontrolle über die Finanzen [vgl. Müller, 2010, S. 5]. Die Unterschicht hatte keine Macht  [vgl. ebd., S. 7].}
% 	\end{addmargin}
% \end{itemize}

% \subsection{Unterschiedliche Zitierweisen}
% \label{sec: ZitierStil}
% Sprechen Sie mit Ihrem*Ihrer Betreuer*in die zu verwendende Zitierweise ab.

% \subsubsection{Quellangaben im Text (AGR, AT, MB, PDK)}
% Meier behauptet, dieser Aspekt sei wichtig~\citeauthoryear[vgl.][S.~5]{Meier:Globalisierung}.\newline
% Meier meint, „dieser Punkt […] ist relevant“~\citeauthoryear[S.~5]{Meier:Globalisierung}.\newline
% Dies entspricht auch der neueren Literatur~\citeauthoryear[vgl.][S.~10]{Mueller:Meier}.\newline
% Vor Kurzem wurde ein Überblick über den aktuellen Forschungsstand zum Thema veröffentlicht~\citeauthoryear[vgl.][S.~20-35]{Mueller:Meier:Huber}.\newline
% In früheren wissenschaftlichen Veröffentlichungen war dieses Thema als Forschungslücke angesehen worden~\citeauthoryear[vgl.][S.~85]{Mueller:Meier:Huber:Tausch}.\newline
% "`Platzhalter-Text für ein wörtliches Zitat aus einem KI-Tool"' \citeauthoryear{OpenAI:2023}.\newline 
% Platzhalter-Text für ein sinngemäßes Zitat aus einem KI-Tool \citeauthoryear[vgl.][]{OpenAI:2023}.\newline
% \underline{BibStyle}: \textsf{FHWelsAlphabeticBrackets}


% \subsubsection{Quellangaben im Text (BI, BUT, IPM, LCW, LTE)}
% Meier behauptet, dieser Aspekt sei wichtig (Meier, 2011, S. 5).\newline
% Meier meint, „dieser Punkt […] ist relevant“ (Meier, 2011, S. 5).\newline
% Dies entspricht auch der aktuelleren Literatur (Müller und Meier, 2019, S. 10).\newline
% Vor Kurzem wurde ein Überblick über den aktuellen Forschungsstand zum Thema veröffentlicht 
% (Müller, Meier und Huber, 2021, S. 20-35).\newline
% In früheren wissenschaftlichen Veröffentlichungen war dieses Thema als Forschungslücke angesehen worden (Müller et al., 2016, S. 85).\newline
% "`Platzhalter-Text für ein wörtliches Zitat aus einem KI-Tool"' (OpenAI, 2023).\newline 
% Platzhalter-Text für ein sinngemäßes Zitat aus einem KI-Tool (OpenAI, 2023).\newline
% \underline{BibStyle}: Diese Zitierweise ist nicht in der Latex-Vorlage abgebildet. Zum Zitieren mit dem
% \verb|\cite|-Befehl sind die Style-Files entsprechend anzupassen!


% \subsubsection{Quellangaben in der Fußnote (AMM, MEWI)}
% Meier behauptet, dieser Aspekt sei wichtig.\footciteauthoryear[Vgl.][S.~5]{Meier:Globalisierung}\newline
% Meier meint, „dieser [\dots] Punkt ist relevant“.\footciteauthoryear[S.~5]{Meier:Globalisierung}\newline
% Dies entspricht auch der aktuelleren Literatur.\footciteauthoryear[Vgl.][S.~10]{Mueller:Meier}\newline
% Vor Kurzem wurde ein Überblick über den aktuellen Forschungsstand zum Thema veröffentlicht.\footciteauthoryear[Vgl.][S.~20-35]{Mueller:Meier:Huber}\newline 
% In früheren wissenschaftlichen Veröffentlichungen war dieses Thema als Forschungslücke angesehen worden.
% \footciteauthoryear[Vgl.][S.~85]{Mueller:Meier:Huber:Tausch}\newline  
% "`Platzhalter-Text für ein wörtliches Zitat aus einem KI-Tool"'. \footciteauthoryear{OpenAI:2023}\newline 
% Platzhalter-Text für ein sinngemäßes Zitat aus einem KI-Tool. \footciteauthoryear[Vgl.][]{OpenAI:2023}\newline
% \underline{BibStyle}: \textsf{FHWelsAlphabeticBrackets}

% \subsubsection{Quellangaben im Text mit Zahlen (AB, AET, AGR, AMM, AT, BUT, EE, LTE, MB, MEWI, RSE, SES, VTP, WFT)}
% Dieser Aspekt ist wichtig~\cite[vgl.][S. 5]{Meier:Globalisierung}.\newline
% Nach \cite[S. 5]{Meier:Globalisierung} ist dieser Aspekt wichtig.\newline
% Meier behauptet, dieser Aspekt sei wichtig~\cite[vgl.][S.~10]{Mueller:Meier}.\newline
% Meier meint, „dieser […] Punkt ist relevant“~\cite[S.~5]{Mueller:Meier}.\newline
% Dies entspricht auch der aktuelleren Literatur~\cite[vgl.][S.~10]{Mueller:Meier:Huber}.\newline
% Vor Kurzem wurde ein Überblick über den aktuellen Forschungsstand zum Thema veröffentlicht~\cite[vgl.][S.~20-35]{Mueller:Meier:Huber:Tausch}.\newline
% In früheren wissenschaftlichen Veröffentlichungen war dieses Thema als Forschungslücke angesehen worden~\cite[vgl.][S.~85]{Mueller:Meier:Huber:Tausch}.\newline
% "`Platzhalter-Text für ein wörtliches Zitat aus einem KI-Tool"'~\cite{OpenAI:2023}.\newline 
% Platzhalter-Text für ein sinngemäßes Zitat aus einem KI-Tool~\cite[Vgl.][]{OpenAI:2023}.\newline
% \underline{BibStyle}: \textsf{FHWelsNumericBrackets}

% \subsubsection{Zitieren mit \LaTeX}
% Die in dieser Vorlage verwendeten Zitierstile können von den Standardstilen \textsf{numeric} und \textsf{authoryear} abgeleitet werden. Dadurch ist es auch möglich, die unterschiedlichen Darstellungsarten zum Zitieren zu verwenden. Genauere Informationen können der Beschreibung des \textit{BibLatex}-Pakets (\url{https://ctan.org/pkg/biblatex?lang=de}) entnommen werden.\newline
% \\Die wichtigsten Befehle:
% \begin{itemize}
% 	\item \verb|\cite{}|:\\ Standardbefehl (\zB: [Meier, 2011])
% 	\item \verb|\cite[pre][post]{}|:\\ erweiterter Standardbefehl (\zB: [<pre> Meier, 2011, <post>])
% 	\item \verb|\citeauthor|:\\ nur Autor der Quelle (\zB: Meier)
% 	\item \verb|\footcite{}|:\\ zitieren als Fußnote
% 	\item \verb|\footcite[pre][post]{}|:\\ zitieren als Fußnote mit Erweiterungen (siehe \verb|\cite[pre][post]{}|)
% \end{itemize}

% \section{Stil}
% \begin{itemize}
% 	\item	Achten Sie beim Schreiben auf eine gendergerechte Sprache (siehe die Satzung der FH OÖ in der geltenden Fassung).
% 	\item	Verwenden Sie stets eine einheitliche Zitierweise. Einheitlichkeit wird als ein wichtiges Merkmal wissenschaftlicher Arbeiten betrachtet.
% 	\item	Die Arbeit ist kein Erlebnisbericht. Vermeiden Sie die Ich-/Wir-Form und ähnliche Formen mit Personalpronomen. Vermeiden Sie im Hinblick auf den angestrebten neutralen Schreibstil außerdem emotional gefärbte Wörter.
% 	\item	Keine Umgangssprache, \zB \textit{Auf einmal entstand \dots}
% 	\item	Achten Sie auf eine korrekte Schreibweise, Grammatik und Zeichensetzung: Mehrgliedrige deutsche Wortzusammensetzungen weisen Bindestriche und keine Leerzeichen auf (\zB Mensch-Maschine-Schnittstelle). Vor \etc sowie \usw steht kein Komma. Bei \zB wird ein Leerzeichen verwendet.
% 	\item	In der indirekten Rede steht der Konjunktiv.
% 	\item	Keine Schlagwortsätze, sondern nur vollständige Sätze mit Verb verwenden (die einzige Ausnahme stellen Strichaufzählungen dar).
% 	\item	Schachtelsätze sind möglichst zu vermeiden.
% 	\item	Nebensätze, die eine Aussage erklären, sind möglichst eigenständige Sätze und nicht irgendwelche Klammersätze.
% 	\item	Klammern sind nur für folgende Aspekte zu nutzen:
% 	\begin{itemize}
% 		\item	Literaturstellen
% 		\item	Verweise auf andere Stellen im Text, wie Tabellen, Bilder \etc
% 		\item	Definition benutzter Abkürzungen
% 	\end{itemize}
	
% 	\item	Bindestriche dienen zur Bildung von Wortzusammensetzungen und zur Silbentrennung. Sie können die gewünschte Silbentrennung eines Wortes am Zeilenende manuell durch die Verwendung eines bedingten Trennstriches steuern.
% 	\item	Verwenden Sie keinen Bindestrich anstelle eines Gedankenstriches.
% 	\item	Am Ende eines Satzes steht maximal ein Punkt, selbst wenn der Satz mit einer Abkürzung endet.
% 	\item	Löschen Sie ungewollt gesetzte doppelte Leerzeichen.
% 	\item	Verwenden Sie zwischen einer Zahl und ihrer Einheit ein geschütztes Leerzeichen (Befehl: \verb|~|), um einen unerwünschten Zeilenumbruch direkt nach der Zahl zu vermeiden.
% 	\item	Wenn unterschiedliche Druckbilder (Fettdruck, Kleingedrucktes \etc) benutzt werden, sollte der Grund dafür klar werden.
% 	\item	Keine Mehrfachhervorhebungen durch Fettdruck und Unterstreichen.
% \end{itemize}

% \section{Rechtliche Aspekte}
% \subsection{Eidestattliche Erklärung}
% Die eidesstattliche Erklärung ist keine reine Formsache, sondern die rechtlich verbindliche Zusicherung, dass alle für die eigene Arbeit verwendeten Materialien angegeben und in der vorgeschriebenen Form im Text entsprechend gekennzeichnet (zitiert) worden sind. Ein erheblicher Verstoß gegen diese Zitiervorschriften bedeutet aus juristischer Sicht den Bruch der eidesstattlichen Erklärung. Eine solche Arbeit muss vom*von der Prüfer*in nach den näheren Vorschriften der jeweiligen Prüfungsordnung abgelehnt werden.\vspace{3mm} \\
% Werden Täuschungsversuche \bzw Betrügereien entdeckt, wird die Arbeit für ungültig erklärt und es muss eine neue Arbeit zu einem neuen Thema verfasst werden. Die Anzahl der möglichen Antritte verringert sich dadurch. \vspace{3mm}\newline
% Ein durch Täuschungen \bzw in betrügerischer Absicht erworbener akademischer Abschluss oder Grad wird im Falle erst zu einem späteren Zeitpunkt bekannt gewordener Verstöße durch die verleihende Institution wieder aberkannt, die Prüfung gilt als nicht bestanden.

% \subsection{Urheberrecht}
% Das Urheberrecht schützt die ideellen Interessen der Autor*inn*en an ihren Werken unter der Voraussetzung, dass es sich um deren persönliche geistige Schöpfung handelt. Die Urheber*innen dürfen alleine darüber bestimmen, ob und auf welche Weise ihr Werk veröffentlicht werden soll (Vervielfältigungs- und Verbreitungsrecht).
% \vspace{3mm}\newline
% Zum Kernpunkt des Urheberrechts gehört, dass es nicht übertragbar ist. Niemand darf sich eine persönliche geistige Schöpfung einer anderen Person zu eigen machen. Der Urheberrechtsschutz muss auch nicht angemeldet werden; er entsteht kraft Gesetzes, \dah jedes Manuskript ist mit seinem Entstehen geschützt.
% \vspace{3mm}\newline
% Bei der Bearbeitung des Themas und der Betreuung der Studierenden sind die Bestimmungen des Urheberrechtsgesetzes, BGBl. Nr. 111/1936, zu beachten.

% \subsection{Umgang mit Markennamen}
% Es besteht keine Rechtspflicht, in wissenschaftlichen Arbeiten Markennamen oder eingetragene Markennamen als solche auszuweisen. In einzelnen Wissenschaftszweigen mag es diesbezüglich jedoch unterschiedliche Empfehlungen geben. Wollen Sie auf (eingetragene) Markennamen hinweisen, die Sie in Ihrer wissenschaftlichen Arbeit verwenden, so können Sie beispielsweise eine Formulierung nach folgendem Muster verwenden:
% \vspace{3mm}\newline
% [Markenname 1 und Markenname 2] sind eingetragene Marken [des Unternehmens X].
% 		\pagestyle{plain}
% 		\eidesErklaerung
% 		\clearpage
% 	\endgroup

	
	\begingroup
		\pagestyle{plain}
		% \include{Text/0002-Kurzfassung}
		% \include{Text/0003-Abstract}
		%
		% Inhaltsverzeichnis:
		\tableofcontents
		\clearpage
	\endgroup
%%	
%%============================================================================
%%            Hauptteil der Arbeit
%%----------------------------------------------------------------------------
	\mainmatter					% arabische Seitennummierierung
	\clearpage
%%----------------------------------------------------------------------------
%% Einleitung	
	\chapter{Einleitung}
\label{sec: Einleitung}

\section{Aufgabenstellung}
Das Hubwerksgetriebe aus dem Skript MMB4 von Dr. Witteeveen soll mit einem Csharp Programm simuliert werden. 
Dazu sind die einzelnen Komponenten des Getriebes zu modellieren und in einem Gesamtmodell zu verknüpfen.
Die Simulation soll es ermöglichen, verschiedene Lastfälle durchzuspielen und die Auswirkungen auf die einzelnen
Komponenten zu beobachten.

\begin{figure}[h]
	\centering
	\includegraphics[width=0.80\textwidth]{/Users/danielweindl/_source/Repositorys/SSI_Hubwerksgetriebe/Bericht/Grafik/1.2.2-Hubwerksgetriebe.png}
	\caption{Aufgabe: Hubwerksgetriebe}
\end{figure}

\section{Aufgabenverteilung}
\begin{itemize}

	\item Visualisierung des Hubwerkgetriebes - DW
	\item Modellierung der Bewegungsgleichung - TP
	
	\item Implementierung der Regelung - TP und DW
	\item Simulation des Gesamtsystems - TP und DW

\end{itemize}

\section{Vorgehensweise}
Die Idee besteht darin, mithilfe einer SFunctionContinuous eine Regelung zu simulieren. Die dabei berechneten Werte der Größe phi sollen anschließend in eine CSV-Datei geschrieben werden, welche in weiterer Folge zur Visualisierung in OpenGL verwendet wird.
%%----------------------------------------------------------------------------
%% Hauptkapitel 1
	\chapter{Visualisierung}
\label{sec: Kapitel 1}


%%----------------------------------------------------------------------------
%% Hauptkapitel 2
	\chapter{Modellbildung}
\label{sec: Kapitel 2}

\section{Berechnung der Bewegungsgleichung}
Die kinetische Energie des Hubwerksgetriebes schreibt sich als
\[
T
= \frac{1}{2} J_2 \dot{\varphi}_2^{\,2}
+ \frac{1}{2} J_3 \dot{\varphi}_3^{\,2}
+ \frac{1}{2} J_4 \dot{\varphi}_4^{\,2}
+ \frac{1}{2} m_5 \dot{y}_5^{\,2}
\]

Aus der Forderung, dass die überstrichenen Bogenlängen beim Abwälzvorgang
für beide involvierten Körper gleich sind, ergibt sich
\[
\varphi_2 r_2 = - \varphi_3 r_{32}
\;\Rightarrow\;
\varphi_3 = - \varphi_2 \frac{r_2}{r_{32}}
\]

\[
\varphi_3 r_{34} = - \varphi_4 r_{43}
\;\Rightarrow\;
\varphi_4 = \varphi_2 \frac{r_2}{r_{32}} \frac{r_{34}}{r_{43}}
\]

\[
y_{55} = \varphi_4 r_{45} - l_0
\;\Rightarrow\;
y_{55} = \varphi_2 \frac{r_2}{r_{32}} \frac{r_{34}}{r_{43}} r_{45} - l_0
\]

Das Einsetzen des Quadrates der zeitlichen Ableitungen in die kinetische
Energie liefert das reduzierte Massenträgheitsmoment \( J_{\text{red}} \).

\[
T
= \frac{1}{2} \dot{\varphi}_2^{\,2}
\left(
J_2
+ J_3 \left( \frac{r_2}{r_{32}} \right)^2
+ J_4 \left( \frac{r_2 r_{34}}{r_{32} r_{43}} \right)^2
+ m_5 \left( \frac{r_2 r_{34} r_{45}}{r_{32} r_{43}} \right)^2
\right)
= \frac{1}{2} \dot{\varphi}_2^{\,2} J_{\text{red}}
\]

Die potentielle Energie ergibt sich zu
\[
V
= m_5 g
\left(
\varphi_2 \frac{r_2 r_{34} r_{45}}{r_{32} r_{43}}
\right)
\]

und unter Anwendung der Lagrange-Gleichung ergibt sich mit dem
Antriebsmoment \( M_2 \) die Bewegungsgleichung
\[
J_{\text{red}} \ddot{\varphi}_2
+ m_5 g \frac{r_2 r_{34} r_{45}}{r_{32} r_{43}}
= M_2
\]

Simulink arbeitet am einfachsten mit expliziten Integrationsketten:
\[
\ddot{\varphi}_2
= \frac{1}{J_{\mathrm{red}}}
\left(
M_2
- m_5 g \frac{r_2 r_{34} r_{45}}{r_{32} r_{43}}
\right)
\]

\section{Blockschaltbild}
\begin{center}
\begin{tikzpicture}[
    >=stealth,
    block/.style={draw, rectangle, minimum width=2.6cm, minimum height=1.2cm, align=center},
    sum/.style={draw, circle, minimum size=1.2cm},
    int/.style={draw, rectangle, minimum width=1.2cm, minimum height=1.2cm}
]

% Knoten
\node[sum]   (sum)  at (0,0) {};
\node[block] (gain) at (3.2,0) {$\dfrac{1}{J_{\mathrm{red}}}$};
\node[int]   (int1) at (6.2,0) {$\displaystyle \int$};
\node[int]   (int2) at (8.6,0) {$\displaystyle \int$};

% Beschriftungen Summationsknoten
\node at (-0.2,0.35) {$+$};
\node at (-0.2,-0.35) {$-$};

% Eingänge
\draw[->] (-2.2,0.6) node[left] {$M_2$} -- (sum.west |- 0,0.6);
\draw[->] (-2.2,-0.6) node[left] {$C_g$} -- (sum.west |- 0,-0.6);

% Hauptpfad
\draw[->] (sum) -- (gain);
\draw[->] (gain) -- (int1);
\draw[->] (int1) -- (int2);
\draw[->] (int2) -- (10.2,0) node[right] {$\varphi_2$};

% Abzweig auf Winkelgeschwindigkeit
\draw[->] (int1.south) |- (6.2,-1.8) -- (10.2,-1.8) node[right] {$\dot{\varphi}_2$};

\end{tikzpicture}
\end{center}


%%----------------------------------------------------------------------------
%% Zusammenfassung
% 	\chapter{Fazit}
\label{sec: Zusammenfassung}

Durch die konsequente Trennung von Berechnung und Visualisierung war es möglich, beide Entwicklungsbereiche parallel zu beginnen und weitgehend unabhängig voneinander weiterzuentwickeln. Als einzige Schnittstelle zwischen diesen beiden Bereichen dient die CSV-Datei, über welche die berechneten Daten an die Visualisierung übergeben werden. Diese klare Trennung erwies sich als vorteilhaft für Struktur, Wartbarkeit und Parallelisierung der Entwicklungsarbeit.

Die Umsetzung stellte sich insgesamt als anspruchsvoller heraus als ursprünglich angenommen. Zwar wurde für die Visualisierung KI-Unterstützung herangezogen, dennoch waren zahlreiche manuelle Anpassungen notwendig. Insbesondere zeigte sich, dass ein grundlegendes Verständnis des erzeugten Codes sowie eine im Vorfeld durchdachte Softwarearchitektur unerlässlich sind, um die Ergebnisse zielgerichtet weiterentwickeln zu können.

Zur besseren Nachvollziehbarkeit und Überprüfung der Rotationsbewegungen wurden an den Rollen gezielt Markierungen angebracht. Dadurch konnten die Bewegungsabläufe visuell überprüft und mit den berechneten Werten abgeglichen werden.

Die Kopplung von Logik und Visualisierung funktionierte insgesamt sehr gut, was maßgeblich darauf zurückzuführen ist, dass bereits im Vorfeld klare konzeptionelle Überlegungen zur Schnittstelle und zum Datenfluss angestellt wurden.

Abschließend lässt sich festhalten, dass das Arbeiten über mehrere Disziplinen hinweg sowie deren Zusammenführung durch Code eine besonders anspruchsvolle, zugleich jedoch sehr interessante und lehrreiche Tätigkeit darstellt.

% %%----------------------------------------------------------------------------
% %% Abkürzungsverzeichnis[optional]
% 	\clearpage
% 	\input{Text/9001-Abkuerzungen}
% %%----------------------------------------------------------------------------
% 	%% Fachwortverzeichnis[optional]
% 	\clearpage
% 	\input{Text/9002-Fachwortverzeichnis}
% %%----------------------------------------------------------------------------
% %% Abbildungsverzeichnis [optional]
% 	\clearpage
% 	\input{Text/9003-Abbildungsverzeichnis}
% %%----------------------------------------------------------------------------
% %% Tabellenverzeichnis [optional]
% 	\clearpage
% 	\input{Text/9004-Tabellenverzeichnis}
% %%----------------------------------------------------------------------------
% %% Formelverzeichnis [optional]
% 	\clearpage
% 	\input{Text/9005-Formelverzeichnis}
% %%----------------------------------------------------------------------------
% %% Literaturverzeichnis
% 	\clearpage
% 	\input{Text/9006-Literaturverzeichnis}
% %%----------------------------------------------------------------------------
% 	%% Anhang[optional]
% 	\input{Text/9010-Anhang}
%%----------------------------------------------------------------------------
%% Messbox zur Druckkontrolle
% 	\input{Text/9999-Messbox}
%%============================================================================	
\end{document}
%%++++++++++++++++++++++++++++++++++++++++++++++++++++++++++++++++++++++++++++